\documentclass[12pt]{article}

\usepackage[brazilian]{babel}
\usepackage[utf8]{inputenc}
\usepackage{graphicx}
\usepackage{mathtools}
\usepackage{amsthm}
\usepackage{amssymb}
\usepackage{thmtools,thm-restate}
\usepackage{amsfonts}
\usepackage{hyperref}
\usepackage[singlelinecheck=false]{caption}
\usepackage[backend=biber,url=true,doi=true,eprint=false,style=numeric]{biblatex}
\usepackage{enumitem}
\usepackage[justification=centering]{caption}
\usepackage{indentfirst}
\usepackage{algorithm}
\usepackage[noend]{algpseudocode}
\usepackage{listings}
\usepackage[x11names,rgb,table]{xcolor}
\usepackage{tikz}
\usepackage{hyperref}
\usepackage{subcaption}
\usepackage{booktabs}
\usepackage{linegoal}
\usepackage{geometry}
\usetikzlibrary{snakes,arrows,shapes}

\addbibresource{references.bib}
\graphicspath{{imgs/}}

\makeatletter
\def\subsection{\@startsection{subsection}{3}%
  \z@{.5\linespacing\@plus.7\linespacing}{.1\linespacing}%
  {\normalfont}}
\makeatother

\makeatletter
\patchcmd{\@setauthors}{\MakeUppercase}{}{}{}
\makeatother

\DeclareMathOperator*{\argmin}{arg\,min}
\DeclareMathOperator*{\argmax}{arg\,max}
\DeclareMathOperator*{\Val}{\text{Val}}
\DeclareMathOperator*{\Ch}{\text{Ch}}
\DeclareMathOperator*{\Pa}{\text{Pa}}
\DeclareMathOperator*{\Sc}{\text{Sc}}
\newcommand{\ov}{\overline}
\newcommand{\tsup}{\textsuperscript}

\newcommand\defeq{\mathrel{\overset{\makebox[0pt]{\mbox{\normalfont\tiny\sffamily def}}}{=}}}

\newcommand{\algorithmautorefname}{Algorithm}
\algrenewcommand\algorithmicrequire{\textbf{Entrada}}
\algrenewcommand\algorithmicensure{\textbf{Saída}}
\algrenewcommand\algorithmicif{\textbf{se}}
\algrenewcommand\algorithmicthen{\textbf{então}}
\algrenewcommand\algorithmicelse{\textbf{senão}}
\algrenewcommand\algorithmicfor{\textbf{para todo}}
\algrenewcommand\algorithmicdo{\textbf{faça}}
\algnewcommand{\LineComment}[1]{\State\,\(\triangleright\) #1}

\captionsetup[table]{labelsep=space}

\theoremstyle{plain}

\newcounter{dummy-def}\numberwithin{dummy-def}{section}
\newtheorem{definition}[dummy-def]{Definition}
\newcounter{dummy-thm}\numberwithin{dummy-thm}{section}
\newtheorem{theorem}[dummy-thm]{Theorem}
\newcounter{dummy-prop}\numberwithin{dummy-prop}{section}
\newtheorem{proposition}[dummy-prop]{Proposition}
\newcounter{dummy-corollary}\numberwithin{dummy-corollary}{section}
\newtheorem{corollary}[dummy-corollary]{Corollary}
\newcounter{dummy-lemma}\numberwithin{dummy-lemma}{section}
\newtheorem{lemma}[dummy-lemma]{Lemma}
\newcounter{dummy-ex}\numberwithin{dummy-ex}{section}
\newtheorem{exercise}[dummy-ex]{Exercise}
\newcounter{dummy-eg}\numberwithin{dummy-eg}{section}
\newtheorem{example}[dummy-eg]{Example}

\numberwithin{equation}{section}

\newcommand{\set}[1]{\mathbf{#1}}
\newcommand{\pr}{\text{P}}
\newcommand{\eps}{\varepsilon}
\newcommand{\ddspn}[2]{\frac{\partial#1}{\partial#2}}
\newcommand{\iddspn}[2]{\partial#1/\partial#2}
\newcommand{\indep}{\perp}
\renewcommand{\implies}{\Rightarrow}

\newcommand{\bigo}{\mathcal{O}}

\setlength{\parskip}{1em}

\lstset{frameround=fttt,
	numbers=left,
	breaklines=true,
	keywordstyle=\bfseries,
	basicstyle=\ttfamily,
}

\newcommand{\code}[1]{\lstinline[mathescape=true]{#1}}
\newcommand{\mcode}[1]{\lstinline[mathescape]!#1!}

\title{%
  Visualizing Generative Sum-Product Networks on Image Completion\\~\\
  {\normalfont Acompanhamento de Projeto MAC6914}
}
\author{Renato Lui Geh, NUSP: 8536030}
\date{}

\begin{document}

\maketitle

\section{Progresso}

O código para aprendizagem e compleição de imagem já está implementado. No entanto, nas últimas
semanas, foram feitas várias atualizações no código para consertar \textit{bugs} de implementação e
de performance. Além disso, para diminuir o tempo de aprendizado, várias partes do código foram
paralelizadas.

\section{A ser feito}

Falta rodar os algoritmos de aprendizagem no conjunto de dados e analizar o escopo das diferentes
camadas da estrutura. Esta análise será feita colorindo os subconjuntos de pixels de cada camada de
uma cor diferente. Também espera-se repetir estes mesmos passos tanto para inferência exata como
aproximada.

Se houver tempo, estas etapas serão repetidas para imagens coloridas RGB.

\section{Estrutura do artigo}

O artigo será dividido nas seções:

\begin{enumerate}
  \item Introduction
  \item Sum-Product Networks
  \item Visualizing SPNs
  \item Conclusions
\end{enumerate}

Na seção \textit{Introduction}, serão revisados artigos atuais na literatura relacionados ao assunto, além
de explicar formalmente compleição de imagem. A segunda seção, \textit{Sum-Product Networks},
descreverá brevemente SPNs, além de apresentar quais algoritmos de aprendizagem de SPNs foram
usados. \textit{Visualizing SPNs} explicará como foi feita a visualização de SPNs. Serão incluídas
imagens dos escopos das diferentes camadas da estrutura, além de outras possíveis visualizações
relevantes. A última seção \textit{Conclusions} será dedicada à discussão e conclusões derivadas do
trabalho desenvolvido.

\section{Estimação de tempo para conclusão}

Espera-se terminar o trabalho até o final de dezembro ou meio de janeiro.

\printbibliography[]

\end{document}
